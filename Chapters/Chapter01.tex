\chapter{Définitions}
\label{Chapter01}

\section{Dénombrement}

\subsection{Cardinal}

Si l'on considère un ensemble $E$ quelconque, le nombre d'éléments distincts qui le composent est appelé \textbf{cardinal de E} et se note $card(E)$.

\begin{example}
    $card(\emptyset) = 0$

    $card(\mathbb{N}) = card(\mathbb{R}) = \infty$

    $card( \{5, 37, \sqrt{2}, 25.54, \pi\} ) = 5$
\end{example}

\subsection{Listes}

Une liste est un ensemble ordonné d'éléments.

\begin{definition}
    Le nombre de listes composées de $k$ éléments chacun compris entre $1$ et $n$ vaut $n^k$.
\end{definition}
\begin{example}
    Si on tire 3 fois un D6 pour créer une liste ordonnée de 3 éléments, il y a $6^3$ possibilités de listes différentes.
\end{example}

\subsection{Arrangements}

Dans un ensemble à $n$ éléments, le nombre de listes de $k$ éléments sans répétition (c'est-à-dire : tous les éléments d'une liste sont uniques) est appelé \textbf{arrangement} et se note $A^k_n$.

\begin{definition}
    L'arrangement de $k$ éléments parmi $n$ vaut :

    \begin{equation*}
        \begin{split}
            \forall (k,n) \in \mathbb{N}^2, k \le n \\
            A^k_n = \frac{n!}{(n-k)!}
        \end{split}
    \end{equation*}
\end{definition}

\subsection{Permutations}

Le nombre de permutations possibles d'un ensemble de cardinal $n$ vaut $n!$.

\begin{remarque}
    Cela correspond à un arrangement de $n$ éléments parmi $n$ :
    \begin{equation*}
        A^n_n = \frac{n!}{(n-n)!} = n!
    \end{equation*}
\end{remarque}

\subsection{Combinaisons}

Le nombre de listes non ordonnées de $k$ éléments uniques parmi $n$ est appelé \textbf{combinaison} et se note $C^k_n$ ou $\binom{n}{k}$. On peut retrouver graphiquement les valeurs de ce coefficient binomial en utilisant le \href{https://en.wikipedia.org/wiki/Pascal%27s_triangle}{triangle de Pascal}.

\begin{definition}
    \begin{equation*}
        \begin{split}
            \forall (k,n) \in \mathbb{N}^2, k \le n \\
            C^k_n = \binom{n}{k} = \frac{n!}{k!(n-k)!}
        \end{split}
    \end{equation*}
\end{definition}

\begin{remarque}
    La combinaison est liée à l'arrangement :
    \begin{equation*}
        C^k_n = \frac{A^k_n}{k!}
    \end{equation*}
\end{remarque}

\begin{remarque}
    Un ensemble de $k$ éléments parmi $n$ a la même taille que son complémentaire :

    \begin{equation*}
        \binom{n}{k} = \binom{n}{n-k}
    \end{equation*}
\end{remarque}

\subsection{Choix d'une méthode de dénombrement}

En cas de doute sur la méthode à choisir, suivre ce diagramme :
\newline


% Define block styles
\tikzstyle{startstop} = [rectangle, rounded corners, minimum width=3cm, minimum height=1cm,text centered, draw=black]%, fill=red!30]
\tikzstyle{process} = [rectangle, minimum width=3cm, minimum height=1cm, text centered, draw=black, fill=orange!30]
\tikzstyle{decision} = [diamond, aspect=3, minimum width=3cm, minimum height=1cm, text centered, draw=black, fill=green!30]
\tikzstyle{arrow} = [thick,->,>=stealth]

\begin{tikzpicture}[node distance=3cm]
	\node (start) [startstop]
		{Départ};
	\node (ordon) [decision, below of=start]
		{Résultat ordonné ?};
	\node (comb)  [startstop, below of=ordon]
		{Combinaison $\binom{n}{k}$};
	\node (repet) [decision, right of=ordon, xshift=4cm] 		{Répétitions ?};
	\node (kuplet)[startstop, right of=repet, xshift=3cm]
		{k-uplet $n^k$};
	\node (arrang)[startstop, below of=repet]
		{Arrangement $A^k_n$};
	\node (permut)[startstop, below of=kuplet]
		{Permutation $n!$};

    \coordinate[below=1cm of repet] (fork);

	\draw [arrow] (start) --                            (ordon);
	\draw [arrow] (ordon) -- node[anchor=south] {oui}   (repet);
	\draw [arrow] (ordon) -- node[anchor=east]  {non}   (comb);
	\draw [arrow] (repet) -- node[anchor=south] {oui}   (kuplet);
	\draw [arrow] (repet) -- node[anchor=east]  {non}   (arrang);
	\draw [arrow] (repet) -- (fork) -| 	                (permut);
\end{tikzpicture}

\section{Probabilités}

\subsection{Expérience aléatoire}

%\begin{definition}
    On appelle \textbf{expérience aléatoire} une expérience qui fait intervenir le hasard. On connaît l’ensemble des \textbf{issues} (ou résultats) possibles sans savoir laquelle de celles-ci se réalisera. Il est possible de répéter un certain nombre de fois cette expérience dans des conditions identiques.
%\end{definition}

\begin{definition}
    L’ensemble, souvent noté $\Omega$, de toutes les issues possibles est appelé \textbf{univers} ou \textbf{espace d’échantillonnage} de l’expérience.
\end{definition}

\begin{example}
    \begin{itemize}
        \item On lance un D6 : $\Omega = \{1,2,3,4,5,6\}$
        \item Dans une chaîne de fabrication, on contrôle l'état des pièces en sortie : $\Omega = \{\text{"conforme", "non conforme"}\}$
        \item On choisit un point dans le plan : $\Omega = \mathbb{R}^2$
    \end{itemize}
\end{example}

\subsection{Évènement}

Un sous-ensemble de $\Omega$ est appelé \textbf{évènement}. Il correspond à la réalisation d'une proposition. Par exemple, si on lance un D6, et que l'on définit la proposition << le résultat est 2 ou 5 >>, alors l'évènement associé est $\{2,5\} \subset \Omega$.

\begin{definition}
    On note $\mathcal{P}(\Omega)$ l'ensemble des évènements de l'expérience.
\end{definition}

\begin{definition}
    \begin{itemize}
        \item $\Omega$ est l'évènement \textbf{certain}
        \item $\emptyset$ est l'évènement \textbf{impossible}
        \item On appelle évènement \textbf{élémentaire} un évènement qui n'a qu'une seule issue, souvent noté $\omega$.
    \end{itemize}
\end{definition}

\begin{example}
    Si on lance un D6, << le résultat est 1 >> est un évènement élémentaire, alors que << on obtient un nombre pair >> est un évènement normal (donc composé d'évènements élémentaires).
\end{example}

\begin{example}
    Si on a $\Omega = \{\omega_1,\omega_2,\omega_3\}$, alors :
    \begin{equation*}
        \mathcal{P}(\Omega) = \{ \emptyset,\{\omega_1\},\{\omega_2\},\{\omega_3\},\{\omega_1,\omega_2\},\{\omega_2,\omega_3\},\{\omega_1,\omega_3\},\{\omega_1,\omega_2,\omega_3\} \}
    \end{equation*}
    \begin{equation*}
        card(\mathcal{P}(\Omega)) = 8 \neq card(\Omega)
    \end{equation*}
\end{example}